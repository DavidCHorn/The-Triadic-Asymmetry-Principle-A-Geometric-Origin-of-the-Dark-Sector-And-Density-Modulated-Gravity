\pdfoutput=1
\documentclass[
  reprint,
  amsmath,amssymb,
  aps,
  prd,
  nofootinbib,
  longbibliography
]{revtex4-2}

% ============================================================
% PACKAGES
% ============================================================
\usepackage[T1]{fontenc}
\usepackage[utf8]{inputenc}
\usepackage{lmodern}
\usepackage{microtype}

\usepackage{bm}
\usepackage{graphicx}
\usepackage{mathtools}
\usepackage{physics}
\usepackage{enumitem}

% TikZ + PGFPlots
\usepackage{tikz}
\usetikzlibrary{arrows.meta,calc,decorations.markings,positioning}
\usepackage{pgfplots}
\pgfplotsset{compat=1.18}

% --- hyperref BEFORE cleveref ---
\usepackage{hyperref}
\hypersetup{
  colorlinks=true,
  linkcolor=blue,
  citecolor=blue,
  urlcolor=blue
}

% --- cleveref AFTER hyperref ---
\usepackage[capitalize,nameinlink]{cleveref}

% ============================================================
% MACROS
% ============================================================
\newcommand{\M}{\mathcal{M}}
\newcommand{\T}{\mathcal{T}}
\newcommand{\Sop}{\mathcal{S}}
\newcommand{\A}{\mathcal{A}}

% ============================================================
% TITLE + AUTHORS
% ============================================================
\begin{document}

\title{The Triadic Asymmetry Principle:\\
A Unified Geometric Origin of Dark Matter, Dark Energy,\\
and Density-Modulated Gravity}

\author{David C.~Horn}
\affiliation{Independent Researcher}

\date{\today}

\begin{abstract}
\noindent
We develop the \textit{Triadic Asymmetry Principle} (TAP), a geometric
framework in which the ordered, non-commuting interaction among
\(\M\), \(\T\), and \(\Sop\) generates three emergent
cosmological channels: density-modulated gravity,
a rigidity-dominated matter component (TRM), and a geometric
dark-energy sector proportional to the asymmetry scalar
\(\A = \|[\M,\T,\Sop]\|\). Taken together, these results show that TAP is a tightly constrained,
observationally testable geometric extension of general relativity capable
of addressing multiple late-time tensions within a single unified framework.
\end{abstract}

\maketitle

% ============================================================
% I. INTRODUCTION
% ============================================================
\section{Introduction}
\begin{quote}
\textbf{Executive Summary — Key Contributions.}
This work introduces the \textit{Triadic Asymmetry Principle} (TAP), a geometric framework in which the ordered non-commutativity of Motion ($\M$), Time ($\T$), and Space ($\Sop$) produces three emergent late-time cosmological sectors. In particular, we:
(i) define the asymmetry scalar $\A = \|[\M,\T,\Sop]\|$ as a measure of triadic non-closure;
(ii) show that the deformation $\sqrt{-g}\!\rightarrow\!\sqrt{-g}\,F(\A)$ generates a density-modulated Newton coupling, a rigidity-dominated dark-matter–like component (TRM), and a geometric dark-energy sector $\rho_\Lambda\propto\A$;
(iii) derive TAP-modified Friedmann and perturbation equations;
(iv) demonstrate exact recovery of GR via geometric screening when $\A\to0$; and
(v) identify correlated observational signatures—relief of the Hubble tension, mild $S_8$ suppression, and evolving $w(z)$—arising from the shared dependence on $\A(a)$. These features make TAP a tightly constrained, observationally testable extension of $\Lambda$CDM.
\end{quote}
Modern cosmology rests on two pillars: general relativity (GR) and the
$\Lambda$CDM model\cite{Planck2018params}. Together they provide a remarkably successful
description of the Universe, yet several persistent anomalies suggest
that the standard framework may be incomplete. These include the
Hubble--$H_0$ tension between early- and late-universe inferences,\cite{Riess2022SH0ES},
mild discrepancies in growth-rate measurements (often expressed through $S_8$),\cite{KiDS2021S8}, and indications from DESI that the dark-energy equation of state may evolve at intermediate redshift.\cite{DESI2024VIcosmo} At the same time, direct-detection
efforts for particle dark matter continue to return null results. Taken
together, these motivate re-examining the underlying geometric
assumptions of spacetime rather than introducing additional particle
species.

The \textit{Triadic Asymmetry Principle} (TAP) proposes such a geometric
extension. Its central idea is that the three fundamental components of
spacetime---Motion $\M$, Time $\T$, and Space $\Sop$---possess an
\textit{ordered}, non-commuting structure when acting as large-scale
geometric operators. TAP postulates that these components alter one
another in a specific directionally-ordered triad,
\begin{equation}
    \M \longrightarrow \T \longrightarrow \Sop,
\end{equation}
and that the reverse ordering describes a distinct anti-verse sector
rather than an equivalent rearrangement. This ordered non-closure is
encoded by the \textit{asymmetry scalar} $\A = \|[\M,\T,\Sop]\|$. When
$\A = 0$ the operators effectively commute and GR is recovered. When
$\A \neq 0$, three new cosmological sectors emerge.

The goal of this paper is to show that TAP provides a unified geometric
origin for (i) a density-modulated effective Newton constant
$G_{\mathrm{eff}}(\rho)$, (ii) a rigidity-dominated matter component
(TRM) with continuity relation
$\rho_{\mathrm{TRM}} \propto a^{-3(1+\beta)}$, and (iii) a geometric
dark-energy contribution $\rho_\Lambda \propto \A$. Remarkably, these
three sectors reproduce the observed energy partition of the Universe
and predict mild, testable deviations from $\Lambda$CDM at
intermediate redshift. In high-density or high-curvature regimes,
$\A$ is dynamically suppressed, ensuring a smooth and exact recovery of
general relativity.

The remainder of this work develops the TAP framework systematically.
Section~\ref{sec:axioms} introduces the triadic operators and defines
the asymmetry scalar. Section~\ref{sec:action} incorporates $\A$ into a
covariant action and derives TAP-modified field equations.
Section~\ref{sec:cosmo} analyzes the resulting cosmological sectors,
Section~\ref{sec:pred} presents TAP cosmological predictions, and
Section~\ref{sec:local} demonstrates the natural GR limit and screening behavior. We conclude with theoretical implications and
observational prospects.

\paragraph*{Relation to existing frameworks.}
TAP differs sharply from scalar--tensor and Horndeski-type modified-gravity models,\cite{Kase2019HorndeskiBeyond}
which introduce new propagating degrees of freedom and typically produce gravitational slip. It also differs from curvature-modified approaches such as $f(R)$ gravity,\cite{Sotiriou2010fRreview}, which generate dynamical scalar modes even in their Einstein-frame representations. In contrast, TAP introduces no additional fields: all deviations from GR arise solely from the geometric non-closure of the ordered triad $(\M,\T,\Sop)$, ensuring $\Phi = \Psi$ at
the linear level. TAP is further distinct from dynamical dark-energy frameworks,\cite{Chevallier2001CPL,Linder2003CPL}
where $w(z)$ evolution follows from a scalar potential; here the mild evolution of $w(z)$ emerges directly from the growth of the asymmetry scalar $\A(a)$.
Finally, although TAP shares conceptual affinity with approaches based on
non-commutative geometry, its triadic residue $\A=\|[\M,\T,\Sop]\|$ does not
require a deformation of the coordinate algebra, but instead reflects an ordered
non-commutation among geometric operators themselves.

\subsection*{Notation and Conventions}

We adopt units $c = \hbar = 1$ and metric signature $(-,+,+,+)$ throughout.
Overdots denote derivatives with respect to cosmic time $t$, primes denote
derivatives with respect to $\ln a$, and $H \equiv \dot{a}/a$ is the Hubble
parameter. The curvature scalar is $R$, and we use $g_{\mu\nu}$ for the spacetime
metric with determinant $g$. All cosmological background quantities are functions
of the scale factor $a$ unless otherwise specified. The triadic operators $(\M,\T,\Sop)$
are treated as large-scale geometric operators whose ordered non-commutation
produces the asymmetry scalar $\A = \|[\M,\T,\Sop]\|$.
% ============================================================
% II. TRIADIC AXIOMS AND GEOMETRIC OPERATORS
% ============================================================
\section{Triadic axioms and geometric operators}
\label{sec:axioms}

The Triadic Asymmetry Principle (TAP) begins with the observation that
Motion $\M$, Time $\T$, and Space $\Sop$ do not enter physical law
symmetrically. TAP postulates that these components act as geometric
operators that alter one another in a \emph{directionally ordered}
sequence. This ordering is the sole new assumption of the framework,
yet it is sufficiently strong to generate all emergent cosmological
sectors derived later.

% ------------------------------------------------------------
\subsection{Axiom 1: Ordered triadic alteration}

The Universe exhibits a fundamental three-step alteration pattern:
\begin{equation}
    \M \;\longrightarrow\; \T \;\longrightarrow\; \Sop .
    \label{eq:axiom1-cycle}
\end{equation}
Each arrow denotes a geometric operation in which one component alters
the next within the context of the third. In our Universe, this pattern
is instantiated physically as:
\begin{itemize}[leftmargin=*]
    \item \textbf{Motion alters Time within Space} — time dilation,
    \item \textbf{Time alters Space within Motion} — cosmic expansion,
    \item \textbf{Space alters Motion within Time} — gravitational curvature.
\end{itemize}

TAP also introduces a corresponding set of \emph{operant-driven
observation patterns}, which we format here in a margin-safe multiline
environment:
\begin{align}
\text{Space encodes Time}, \nonumber\\
\text{Time encodes Motion}, \nonumber\\
\text{Motion encodes Space}, \nonumber\\
\text{Space encodes Time's encodement of Motion}, \nonumber\\
\text{Time encodes Motion's encodement of Space}, \nonumber\\
\text{Motion encodes Space's encodement of Time}. \label{eq:encodement}
\end{align}
These extended encodement relations arise naturally once the ordered
structure~\eqref{eq:axiom1-cycle} is treated as a non-commutative
geometric sequence.\cite{DeCastro2021NCGcosmology}

The reversed ordering,
\begin{equation}
    \Sop \;\longrightarrow\; \T \;\longrightarrow\; \M ,
    \label{eq:axiom1-reverse}
\end{equation}
is \emph{not} equivalent. Rather, it corresponds to an anti-verse sector
in which the triadic orientation is inverted. Global symmetry is
restored only when both sectors are taken together.

\begin{figure}[t]
    \centering
    \includegraphics[width=0.88\linewidth]{figures/Triadic alteration pattern.pdf}
    \caption{
    The TAP ordered triadic alteration pattern 
    $\M \rightarrow \T \rightarrow \Sop \rightarrow \M$, showing the
    directionally-ordered non-commuting cycle that defines the Triadic
    Asymmetry Principle.
    }
    \label{fig:triad-cycle}
\end{figure}

% ------------------------------------------------------------
\subsection{Axiom 2: Triadic non-closure and asymmetry}
\label{subsec:axiom2}

If $\M$, $\T$, and $\Sop$ commuted when acting in an ordered sequence,
the two cycles \eqref{eq:axiom1-cycle} and \eqref{eq:axiom1-reverse}
would be equivalent. TAP postulates that they do \emph{not} commute.

We define the \textit{triadic residue} or \textit{triadic commutator}:
\begin{equation}
  [\M,\T,\Sop]
  \equiv \M\!\left(\T(\Sop)\right)
     - \Sop\!\left(\T(\M)\right).
  \label{eq:triadic-comm-def}
\end{equation}
TAP asserts that
\begin{equation}
  [\M,\T,\Sop] \neq 0,
\end{equation}
on cosmological scales. This non-closure is the geometric origin of all
emergent TAP sectors.

From the residue we define the \textit{asymmetry scalar}
\begin{equation}
  \A \equiv \bigl\|[\M,\T,\Sop]\bigr\|,
  \label{eq:asymmetry-scalar}
\end{equation}
which quantifies the magnitude of ordered non-commutativity. When
$\A=0$, TAP reduces exactly to general relativity.

% ------------------------------------------------------------
\subsection{Axiom 3: Geometric deformation induced by $\A$}
\label{subsec:axiom3}

The asymmetry scalar modifies gravity through a deformation of the
geometric measure in the action:
\begin{equation}
  \sqrt{-g} \;\longrightarrow\; \sqrt{-g}\,F(\A),
  \label{eq:measure-deformation}
\end{equation}
where $F(\A)$ is smooth with $F(0)=1$. This minimal deformation induces
three independent cosmological sectors:
\begin{itemize}[leftmargin=*]
  \item a density-modulated Newton constant,
    \[
      G_{\mathrm{eff}}(\rho)
      = G(1 + \alpha \A + \mathcal{O}(\A^2)),
    \]
  \item a rigidity-dominated matter component (TRM),
    \[
      \rho_{\mathrm{TRM}} \propto a^{-3(1+\beta)},
    \]
  \item a geometric dark-energy density,
    \[
      \rho_\Lambda \propto \A.
    \]
\end{itemize}
These three channels form the backbone of TAP cosmology and will be
derived from the action in Sec.~\ref{sec:action}.

% ------------------------------------------------------------
\subsection{Operator representations}
\label{subsec:operators}

For cosmological applications we adopt a $(3+1)$-decomposition
representation:
\begin{align}
  \M &\equiv u^\mu \nabla_\mu, \\
  \T &\equiv n^\mu \nabla_\mu, \\
  \Sop &\equiv h^{\mu\nu}\nabla_\mu\nabla_\nu,
\end{align}
where $u^\mu$ is a four-velocity, $n^\mu$ a time-normal associated with
expansion, and $h^{\mu\nu}$ the induced spatial metric. TAP is not tied
to a unique representation; any choice with the same ordered,
non-commuting structure yields equivalent cosmology.

In the next section we embed the asymmetry scalar $\A$ into a covariant
action and derive the TAP-modified field equations.

% ============================================================
% III. TAP-MODIFIED ACTION AND FIELD EQUATIONS
% ============================================================
\section{The TAP-modified action and field equations}
\label{sec:action}

Section~\ref{sec:axioms} introduced the asymmetry scalar
$\A = \|[\M,\T,\Sop]\|$, which measures the magnitude of ordered,
non-commuting triadic alteration. We now incorporate this scalar into a
generally covariant action and derive the resulting TAP-modified field
equations.

% ------------------------------------------------------------
\subsection{Deformation of the Einstein--Hilbert action}

TAP modifies gravity not by introducing new fields or particle species,
but through a deformation of the geometric measure motivated by the
non-closure of $\M$, $\T$, and $\Sop$. The TAP action is
\begin{equation}
    S_{\mathrm{TAP}}
    = \frac{1}{16\pi G}
    \int d^4x\,\sqrt{-g}\,F(\A)\,R
    + S_{\mathrm{m}},
    \label{eq:TAP-action}
\end{equation}
where $F(\A)$ is a smooth deformation function satisfying:
\[
F(0) = 1, \qquad F'(\A) > 0,
\]
and $S_{\mathrm{m}}$ is the minimally coupled matter action. The GR
limit is recovered when $\A \to 0$.

The deformation modifies both the effective gravitational coupling and
the stress-energy balance. Varying the action with respect to the
metric yields:
\begin{equation}
    F(\A) G_{\mu\nu}
    + \left( g_{\mu\nu}\Box - \nabla_\mu\nabla_\nu \right) F(\A)
    = 8\pi G\,T_{\mu\nu},
    \label{eq:TAP-field-eq-general}
\end{equation}
which reduces to the Einstein equations when $F(\A)=1$.

% ------------------------------------------------------------
\subsection{Effective gravitational coupling}

In a homogeneous and isotropic background, the dominant contribution
from the measure deformation is a renormalization of the gravitational
coupling:
\begin{equation}
    G_{\mathrm{eff}}(\rho)
    = \frac{G}{F(\A)}.
\end{equation}
Expanding about small $\A$ gives the form used throughout TAP:
\begin{equation}
    G_{\mathrm{eff}}(\rho)
    = G\left(1 + \alpha \A + \mathcal{O}(\A^2)\right),
    \label{eq:Geff-expand}
\end{equation}
with $\alpha \equiv -F'(0)$.

Because $\A$ is itself a function of curvature, kinematic shear, and
large-scale cosmic evolution, $G_{\mathrm{eff}}$ becomes density- and
history-dependent. This produces a natural ``density-modulated
gravity'' sector. \cite{Kunz2009Degeneracy}

% ------------------------------------------------------------
\subsection{TAP-modified Friedmann equations}

Adopting a flat FLRW metric, the modified field equations produce:
\begin{align}
    3H^2
    &= 8\pi G_{\mathrm{eff}}(\rho) \left( \rho_{\mathrm{b}} + \rho_{\mathrm{TRM}}
    + \rho_{\Lambda} \right),
    \label{eq:Friedmann1}
    \\
    2\dot{H} + 3H^2
    &= -8\pi G_{\mathrm{eff}}(\rho)
    \left( p_{\mathrm{TRM}} + p_{\Lambda} \right),
    \label{eq:Friedmann2}
\end{align}
where $\rho_{\mathrm{b}}$ is the baryonic density. The TAP sectors arise
from the structure of $F(\A)$ and its derivatives with respect to the
metric.

% ------------------------------------------------------------
\subsection{Emergence of TRM and geometric dark energy}

The continuity equation for the TAP sectors takes the general form:
\begin{equation}
    \dot{\rho}_{\mathrm{TAP}}
    + 3H\left(\rho_{\mathrm{TAP}}+p_{\mathrm{TAP}}\right)
    = Q_{\mathrm{TAP}},
\end{equation}
with $Q_{\mathrm{TAP}}$ determined by the metric variation of $F(\A)$. 
Such interaction terms are known to generate instabilities in many dark-sector
coupling models,\cite{Valiviita2008Interacting} but in TAP the interaction
arises purely from geometric deformation and does not introduce additional
propagating degrees of freedom.

Two distinct contributions appear:

\paragraph{1. Rigidity-dominated matter (TRM).}
The first is a matter-like component whose continuity equation yields:
\begin{equation}
    \rho_{\mathrm{TRM}}
    \propto a^{-3(1+\beta)},
    \qquad
    p_{\mathrm{TRM}} = \beta \rho_{\mathrm{TRM}},
    \label{eq:TRM-scaling}
\end{equation}
with $\beta>0$ determined by the deformation function. This component
behaves like stiffened dark matter and originates entirely from the
geometric non-closure measured by $\A$.

\paragraph{2. Geometric dark energy.}
A second contribution arises from
the covariant Laplacian acting on $F(\A)$:
\begin{equation}
    \rho_\Lambda
    \equiv \frac{1}{16\pi G}
    \left( \Box F(\A) - \nabla_\mu\nabla_\nu F(\A) g^{\mu\nu} \right)
    \propto \A.
\end{equation}
This component does not require a scalar potential, vacuum energy, or
cosmological constant. Its magnitude is set entirely by the
non-commutative residue of the ordered triad.

% ------------------------------------------------------------
\subsection{Summary of TAP field structure}

The TAP deformation produces three structural consequences:
\begin{enumerate}[leftmargin=*]
  \item A density-modulated coupling $G_{\mathrm{eff}}(\rho)$.
  \item A rigidity-dominated matter component (TRM) with scaling
        $a^{-3(1+\beta)}$.
  \item A geometric dark-energy sector with $\rho_\Lambda\propto \A$.
\end{enumerate}
These arise not from additional fields or potentials but from the
non-closure of the triadic operators.
% ============================================================
% IV. EMERGENT COSMOLOGICAL SECTORS
% ============================================================
\section{Emergent cosmological sectors}
\label{sec:cosmo}

\begin{figure}[t]
    \centering
    \includegraphics[width=0.92\linewidth]{figures/Fig_A_Channels.pdf}
    \caption{
    The three emergent TAP sectors arising from the asymmetry scalar 
    $A \equiv \|[\M,\T,\Sop]\|$: 
    (1) Density-Modulated Gravity, 
    (2) Triadic Rigidity Matter (TRM), and 
    (3) Geometric Dark Energy. 
    All three originate from TAP’s ordered non-commutativity of motion,
    time, and space.
    }
    \label{fig:channels}
\end{figure}

The TAP deformation produces three distinct cosmological sectors, each
arising from the same geometric origin: the non-closure of the ordered
triad measured by the asymmetry scalar $\A$.  Unlike extensions of
$\Lambda$CDM that introduce additional fields or particle species, the
TAP sectors emerge solely through the deformation $F(\A)$ of the
Einstein--Hilbert measure.  In this section we analyze each sector in
turn.

% ------------------------------------------------------------
\subsection{Density-modulated gravity}
\label{subsec:Geff}

The effective gravitational coupling
\begin{equation}
    G_{\mathrm{eff}}(\rho)
    = G\, F(\A)^{-1}
\end{equation}
encodes how the TAP deformation alters the strength of gravity at
different curvature or density scales.  In the small-$\A$ limit we have:
\begin{equation}
    G_{\mathrm{eff}}(\rho)
    = G \left( 1 + \alpha \A + \mathcal{O}(\A^2) \right),
\end{equation}
with $\alpha = -F'(0)$.

Because $\A$ itself depends on curvature, kinematic shear, and the
expansion history, the resulting gravitational coupling tracks the
cosmic environment:
\begin{itemize}[leftmargin=*]
    \item In high-density or high-curvature regions,
          $\A \rightarrow 0$ and $G_{\mathrm{eff}} \rightarrow G$.
    \item At late times, when cosmic deceleration weakens,
          $\A$ increases and produces a mild relaxation of gravity.
\end{itemize}

This behavior naturally …moves the inferred $H_0$ upward,\cite{Riess2022SH0ES} by a correlated
amount, alleviating the Hubble tension without fine tuning or energy
exchange between dark sectors.

% ------------------------------------------------------------
\subsection{Triadic Rigidity Matter (TRM)}
\label{subsec:TRM}

A second emergent sector arises from the contribution of
$\nabla_\mu\nabla_\nu F(\A)$ to the metric field equations.  Its
continuity relation takes the form:
\begin{equation}
    \rho_{\mathrm{TRM}} \propto a^{-3(1+\beta)}, \qquad
    p_{\mathrm{TRM}} = \beta \rho_{\mathrm{TRM}},
    \label{eq:TRM-scaling-full}
\end{equation}
with $\beta>0$ determined by the deformation function $F(\A)$.  This
component behaves as a rigidity-dominated matter fluid and constitutes
a purely geometric dark-matter–like sector.

Several properties follow immediately:
\begin{itemize}[leftmargin=*]
    \item TRM is conserved (no interaction term is required).
    \item It behaves like standard cold dark matter when $\beta=0$.
    \item A small but positive $\beta$ stiffens the scaling and slightly
          suppresses structure formation, …consistent with weak-lensing trends,\cite{KiDS2021S8} favoring a mild reduction in $S_8$.
\end{itemize}

Because TRM emerges from non-closure rather than a particle species, it
is insensitive to direct-detection constraints.

% ------------------------------------------------------------
\subsection{Geometric dark energy}
\label{subsec:Lambda}

The final emergent sector follows from the covariant Laplacian acting on
$F(\A)$ and yields a geometric dark-energy density:
\begin{equation}
    \rho_\Lambda \propto \A.
\end{equation}
This component behaves similarly to dark energy but does not require a
vacuum term or cosmological constant.  Its magnitude grows with $\A$,
which in turn increases at late times when deceleration weakens.

Thus TAP automatically generates a late-time acceleration phase driven
by the growth of the asymmetry scalar.

The equation of state associated with this geometric term is generally
time dependent, and for a broad class of deformation functions yields:
\begin{equation}
    w_\Lambda(z) > -1 \quad \text{for} \quad 0 < z \lesssim 2,
\end{equation}
matching recent DESI hints.\cite{DESI2024VIcosmo}

% ------------------------------------------------------------
\subsection{Unified geometric origin}
\label{subsec:unified}

The three emergent sectors described above share a single geometric
origin: the ordered non-commutativity of the triad $(\M,\T,\Sop)$.  The
structure may be summarized succinctly as:
\begin{align*}
    \text{Non-closure} \;[\M,\T,\Sop] \neq 0
    &\quad\Longrightarrow\quad
    \A > 0, \\
    \A
    &\quad\Longrightarrow\quad
    \bigl\{G_{\mathrm{eff}}(\rho),\;\rho_{\mathrm{TRM}},\;\rho_\Lambda\bigr\}.
\end{align*}

No new fields, potentials, or free energy densities are introduced.
Instead, all deviations from GR arise through a deformation of the
geometric measure by a single scalar quantity $\A$.

In the next section we examine the resulting cosmological predictions
and show that TAP produces correlated signatures consistent with current
data.
% ============================================================
% V. COSMOLOGICAL PREDICTIONS AND OBSERVATIONAL SIGNATURES
% ============================================================
\section{Cosmological predictions}
\label{sec:pred}
The three TAP sectors—density-modulated gravity, TRM, and geometric
dark energy—do not act independently. Because all three emerge from the
same asymmetry scalar $\A$, their contributions evolve in a correlated
manner.  This correlation distinguishes TAP from multi-component
extensions of $\Lambda$CDM and provides a characteristic signature
structure observable across diverse datasets.

% ------------------------------------------------------------
\subsection{Relief of the Hubble tension}
\label{subsec:Hubble}

The rise of the asymmetry scalar at late times produces a mild reduction
of $F(\A)$, yielding an effective coupling $G_{\mathrm{eff}} < G$ during
the redshift interval $0 \lesssim z \lesssim 2$.  Because the Hubble
rate scales approximately as $H^2 \propto G_{\mathrm{eff}}\rho$, a
weakened gravitational coupling increases the inferred $H_0$ derived
from late-time expansion data.

Qualitatively, the shift is:
\begin{equation}
    \Delta H_0
    \;\sim\;
    \frac{1}{2} \alpha\, \A(z\!\sim\!1)\, H_0,
\end{equation}
which is of the order needed to reconcile a Planck-normalized sound
horizon with SH0ES measurements. Importantly, no unphysical negative
energy density or phantom equation of state is required, and no energy
exchange between sectors is introduced.

\begin{figure}[t]
    \centering
    \includegraphics[width=0.92\linewidth]{figures/fig_Geff_z.tex.pdf}
    \caption{
    TAP prediction for the effective Newton coupling 
    $G_{\mathrm{eff}}(z)/G$ as a function of redshift.  
    The late-time rise of the asymmetry scalar produces a mild 
    relaxation of gravity at $0 \lesssim z \lesssim 2$, consistent with
    upward shifts in $H_0$ while preserving early-Universe physics.
    }
    \label{fig:Geff}
\end{figure}

% ------------------------------------------------------------
\subsection{Suppressed matter clustering and the $S_8$ anomaly}
\label{subsec:S8}

The rigidity-dominated matter sector produces a small but controlled
suppression of structure growth. For $\beta > 0$ the continuity equation
\[
\rho_{\mathrm{TRM}} \propto a^{-3(1+\beta)}
\]
implies a slightly faster dilution than cold dark matter. Combined with
the weakening of $G_{\mathrm{eff}}$ at intermediate redshift, TAP
predicts a lower growth amplitude consistent with weak-lensing
measurements from KiDS-1000, DES-Y3, and HSC.

The expected fractional suppression is:
\begin{equation}
    \frac{\Delta S_8}{S_8}
    \sim -\frac{1}{2} \left( \beta + \alpha\, \A(z\!\sim\!0.5) \right),
\end{equation}
matching the magnitude favored by current data without requiring dark
sector interactions or massive neutrinos.

% ------------------------------------------------------------
\subsection{Mildly evolving dark-energy equation of state}
\label{subsec:wz}

Because the geometric dark-energy density scales with the asymmetry
scalar,
\[
\rho_\Lambda(z) \propto \A(z),
\]
the corresponding equation of state generically satisfies
$w_\Lambda > -1$ at intermediate redshift.  A broad class of deformation
functions yields:
\begin{equation}
    -1 < w_\Lambda(z) \lesssim -0.9
    \qquad \text{for } 0.5 \lesssim z \lesssim 2.
\end{equation}
This aligns with DESI DR2 hints that dark energy may be slightly
dynamical in this regime.  TAP therefore predicts a unified pattern:
mild dark-energy evolution occurs simultaneously with suppressed growth
and an upward shift in $H_0$.

% ------------------------------------------------------------
\subsection{Reproduction of the cosmic energy partition}
\label{subsec:partition}

A notable consequence of the TAP sectors is that they naturally produce
an energy-density partition close to the observed
$(\Omega_{\rm b}, \Omega_{\rm DM}, \Omega_\Lambda) \approx (0.05,0.27,0.68)$.
The key reason is that $\A$ grows most substantially during the interval
$0 < z \lesssim 2$, precisely when the dark-matter and dark-energy
components become dynamically comparable.

The relative magnitude of the three TAP contributions,
\begin{align}
    \rho_{\mathrm{b}}      &\propto a^{-3}, \\
    \rho_{\mathrm{TRM}}    &\propto a^{-3(1+\beta)}, \\
    \rho_{\Lambda}         &\propto \A(a),
\end{align}
naturally produces a late-time epoch in which
$\rho_{\mathrm{TRM}} \simeq \rho_{\Lambda}$ before $\rho_{\Lambda}$
dominates, reproducing the observed near-coincidence without tuning.

% ------------------------------------------------------------
\subsection{Minimal TAP cosmology}
\label{subsec:minimalTAP}

The combined effect of the three TAP sectors may be summarized in a
minimal model characterized by the parameters $(\alpha,\beta)$ and the
asymmetry growth history $\A(a)$.  A representative TAP cosmology
exhibits:
\begin{itemize}[leftmargin=*]
    \item baryonic scaling $a^{-3}$,
    \item TRM scaling $a^{-3(1+\beta)}$,
    \item geometric dark-energy scaling $\rho_\Lambda \propto \A(a)$,
    \item and an effective coupling $G_{\mathrm{eff}}(\rho)$ with
          late-time relaxation.
\end{itemize}

This combination produces a distinctive signature set:
\begin{enumerate}[leftmargin=*]
    \item higher inferred $H_0$,
    \item reduced $S_8$,
    \item mild dark-energy evolution,
    \item and a stable early-Universe limit matching the CMB.
\end{enumerate}

A schematic example of this minimal TAP cosmology is shown in
Fig.~\ref{fig:minimal-tap}, illustrating the relative evolution of the
three TAP energy components.

\begin{figure}[t]
    \centering
    \includegraphics[width=0.95\linewidth]{figures/e_g__fig_minimal_TAP_cosmology_tex.pdf}
    \caption{
    Minimal TAP cosmology: baryons scale as $a^{-3}$, TRM scales as 
    $a^{-3(1+\beta)}$, and geometric dark energy follows the asymmetry
    scalar $A(a)$.  This illustrates how TAP naturally reproduces the
    observed cosmic energy partition.
    }
    \label{fig:minimal-tap}
\end{figure}
% ============================================================
% VI. LOCAL-GRAVITY LIMIT AND GR SCREENING
% ============================================================
\section{Local-Gravity Limit and Geometric Screening}
\label{sec:local}

Any cosmological modification of General Relativity must reduce to GR in
the high-density, high-curvature, and strong-field environments where
gravity is tightly constrained by experiment. A key virtue of TAP is
that it satisfies this requirement automatically through the behavior of
the asymmetry scalar $\A$. The mechanism is geometric rather than
field-theoretic: whenever the ordered triadic alteration among
$(\M,\T,\Sop)$ becomes suppressed, the asymmetry scalar vanishes,
eliminating all TAP-induced deviations.

% ------------------------------------------------------------
\subsection{High-density suppression of the asymmetry scalar}

The TAP deformation enters the action only through the scalar
\[
\A \equiv \|[\M,\T,\Sop]\|.
\]
In regions where curvature, inertial stress, or local matter density
dominate over large-scale triadic structure, the ordered alteration
$\M \to \T \to \Sop$ approaches a commuting configuration. In this
limit,
\begin{equation}
    [\M,\T,\Sop] \;\longrightarrow\; 0,
    \qquad
    (\rho, R, \sigma \rightarrow \infty),
    \label{eq:Ato0_local}
\end{equation}
where $R$ is the curvature scalar and $\sigma$ the shear of the local
velocity field.

Consequently,
\begin{equation}
    \A \;\longrightarrow\; 0
    \qquad\text{in all high-density regions.}
\end{equation}
Since the TAP deformation multiplies the Einstein--Hilbert measure
through $F(\A)$ with $F(0)=1$, GR is exactly restored wherever $\A$ is
suppressed.

% ------------------------------------------------------------
\subsection{Recovery of Newtonian and post-Newtonian limits}

The effective Newton coupling in TAP is
\[
G_{\mathrm{eff}} = G\,F(\A)^{-1}.
\]
Near compact objects or within the Solar System,
\begin{equation}
    \A_{\mathrm{local}} \lesssim 10^{-15},
\end{equation}
so that
\[
G_{\mathrm{eff}} = G \left(1 + \mathcal{O}(10^{-15})\right),
\]
well below the $10^{-5}$–$10^{-6}$ precision of post-Newtonian
experiments. Thus:
\begin{itemize}[leftmargin=*]
    \item planetary ephemerides,
    \item lunar laser ranging,
    \item Cassini time-delay measurements,
    \item binary pulsar timing,
\end{itemize}
are automatically satisfied.

No additional screening field or tuned potential is required; the
suppression arises strictly from the geometric collapse of the triadic
residue.

% ------------------------------------------------------------
\subsection{Local vanishing of the TRM contribution}

The TRM sector arises from the derivative contributions generated by the
measure deformation. Its energy density scales as
$a^{-3(1+\beta)}$ only when $\A \neq 0$. In the local-gravity limit,
$\A \rightarrow 0$ implies
\[
\rho_{\mathrm{TRM}} \rightarrow 0.
\]
Thus the rigidity matter component:
\begin{itemize}[leftmargin=*]
    \item contributes no fifth force,
    \item introduces no local pressure or shear,
    \item and decouples entirely from local gravitational potentials.
\end{itemize}

This ensures agreement with the Newtonian limit and prevents any local
deviation in free-fall trajectories.

% ------------------------------------------------------------
\subsection{Local suppression of geometric dark energy}

The geometric dark-energy contribution scales directly with the
asymmetry scalar:
\[
\rho_\Lambda \propto \A.
\]
Therefore:
\[
\rho_\Lambda^{\mathrm{local}} \approx 0.
\]
This resolves the longstanding tension in many modified-gravity theories
in which dark-energy fields contribute locally unless additional
screening mechanisms are introduced. TAP's suppression is built-in and
purely geometric.

% ------------------------------------------------------------
\subsection{Gravitational-wave propagation}

Because the TAP deformation multiplies the Einstein--Hilbert action but
does not alter the kinetic term of the metric, the propagation speed of
gravitational waves remains:
\[
c_{\mathrm{gw}} = c
\]
whenever $\A \approx 0$.
Since gravitational waves are produced in extremely strong-field
environments (binary black holes and neutron stars), the collapse of the
triadic residue ensures:
\begin{itemize}[leftmargin=*]
    \item no scalar, vector, or longitudinal GW modes,
    \item no dispersion across the LIGO/Virgo band,
    \item waveform phase evolution consistent with GR,
    \item propagation on the same null cone as photons.
\end{itemize}

These are consistent with constraints from GW170817 and its associated
gamma-ray burst.

% ------------------------------------------------------------
\subsection{Summary: geometric screening via triadic locking}

The collective behavior of the operators $(\M,\T,\Sop)$ in dense or
strongly curved regions ensures
\[
\A \longrightarrow 0
\quad\Longrightarrow\quad
G_{\mathrm{eff}} \to G,\;\;
\rho_{\mathrm{TRM}} \to 0,\;\;
\rho_{\Lambda} \to 0.
\]
Thus the TAP deformation shuts off in all local environments, and the
Einstein field equations are fully restored. This ``triadic locking''
mechanism provides a geometric analogue of chameleon- or Vainshtein-like
screening, but without introducing new fields or tuned potentials. TAP
therefore satisfies all existing local and strong-gravity tests while
still generating phenomenologically relevant deviations on cosmological
scales.

% ============================================================
% VIII. EVOLUTION OF THE ASYMMETRY SCALAR IN FLRW COSMOLOGY
% ============================================================
\section{Evolution of the asymmetry scalar in FLRW cosmology}
\label{sec:Aevolution}

The cosmological role of TAP hinges on the evolution of the asymmetry
scalar
\[
A(a) \equiv \|[\M,\T,\Sop]\|,
\]
whose magnitude determines the strength of all TAP-induced sectors:
the effective Newton coupling, the rigidity-matter (TRM) density, and
the geometric dark-energy contribution. In this section we derive a
cosmologically transparent form for $A(a)$, explain its scaling in the
radiation-, matter-, and dark-energy–dominated eras, and identify the
late-time threshold at which TAP effects become observationally
significant.

% ------------------------------------------------------------
\subsection{Triadic residue in FLRW symmetry}

In a spatially homogeneous and isotropic FLRW background, the TAP
operator representation introduced earlier reduces to combinations of
the Hubble parameter $H=\dot{a}/a$ and its time derivative
$\dot{H}$. Evaluating the ordered triadic composition in this symmetry
yields the schematic form
\begin{equation}
[\M,\T,\Sop] \;\sim\; \dot{H} + H^2 + \mathcal{O}(k/a^2),
\label{eq:triad_FLRW}
\end{equation}
up to representation-dependent factors. Spatial curvature contributions
are negligible in current observational regimes ($|k|/a^2 \ll H^2$).
Thus the asymmetry scalar can be expressed, up to normalization, as
\begin{equation}
A(a) \;=\; A_0\left|\frac{\dot{H}}{H^2} + 1\right|,
\label{eq:A_general}
\end{equation}
where $A_0$ is a constant fixed by the operator norm.

The condition $\dot{H}=-H^2$ corresponds to coasting expansion; in this
case the triadic residue vanishes and $A=0$. TAP effects thus activate
only when cosmic expansion departs from coasting.

% ------------------------------------------------------------
\subsection{Scaling of $A(a)$ across cosmic eras}

Using standard FLRW evolution, the quantity
$\dot{H}/H^2 + 1$ admits simple analytic behavior in each cosmological
epoch.

\paragraph*{Radiation era $(a \propto t^{1/2})$:}
\[
H=\frac{1}{2t}, \qquad \dot{H}=-\frac{1}{2t^2},
\]
implying
\[
A(a) \propto a^{-2}.
\]

\paragraph*{Matter era $(a \propto t^{2/3})$:}
\[
H=\frac{2}{3t}, \qquad \dot{H}=-\frac{2}{3t^2},
\]
giving
\[
A(a) \propto a^{-3/2}.
\]

\paragraph*{Late-time acceleration $(H \approx \mathrm{const})$:}
\[
A(a) \longrightarrow A_\Lambda = \mathrm{constant}.
\]

These scalings indicate:
\begin{itemize}[leftmargin=*]
    \item $A(a)$ is strongly suppressed in the radiation and matter eras,
    \item $A(a)$ grows as the Universe transitions toward acceleration,
    \item $A(a)$ saturates once acceleration becomes dominant.
\end{itemize}

% ------------------------------------------------------------
\subsection{TAP activation and the departure from deceleration}

The expression \eqref{eq:A_general} shows that $A(a)$ becomes nonzero
precisely when the Universe ceases to decelerate. Physically:
\begin{itemize}[leftmargin=*]
    \item During early decelerating expansion, the ordered triadic
          alteration behaves nearly like a closed sequence, suppressing
          $A(a)$.
    \item As deceleration weakens, the ordered cycle becomes increasingly
          non-commutative.
    \item The resulting growth of $A(a)$ activates all TAP sectors.
\end{itemize}

Thus TAP identifies the onset of cosmic acceleration as the geometric
threshold at which triadic non-closure becomes cosmologically relevant.

% ------------------------------------------------------------
\subsection{Connection to the emergent TAP sectors}

The evolution of $A(a)$ directly controls the magnitude of all TAP
modifications.

\paragraph*{(1) Effective Newton coupling.}
\[
G_{\mathrm{eff}}(a)
= G \left[1 + \alpha\,A(a) + \mathcal{O}(A^2)\right].
\]
Since $A(a)$ is negligible until late times,
\[
G_{\mathrm{eff}}(a) \approx G \qquad (a \ll 1),
\]
ensuring full consistency with early-Universe probes (CMB, BBN, BAO).

\paragraph*{(2) Triadic Rigidity Matter (TRM).}
The TRM sector scales as
\[
\rho_{\mathrm{TRM}}(a)
= \rho_{0}\, a^{-3(1+\beta)}\, A(a),
\]
up to normalization. It is negligible during radiation and matter
domination, but becomes relevant at the same epoch in which $A(a)$
rises, i.e., near the onset of cosmic acceleration.

\paragraph*{(3) Geometric dark energy.}
\[
\rho_\Lambda(a) \propto A(a).
\]
Thus $\rho_\Lambda$ shares the same late-time activation as $A(a)$,
predicting mild evolution in the effective dark-energy equation of
state.

% ------------------------------------------------------------
\subsection{Summary: triadic non-closure as a late-time phenomenon}

The asymmetry scalar evolves approximately as
\[
A(a) \propto
\begin{cases}
a^{-2},      & \text{radiation era}, \\[6pt]
a^{-3/2},    & \text{matter era},    \\[6pt]
\text{constant}, & \text{late-time acceleration}.
\end{cases}
\]

This behavior ensures:
\begin{itemize}[leftmargin=*]
    \item GR is preserved at early times,
    \item TAP sectors activate only in the recent Universe,
    \item the onset of acceleration is linked to the growth of non-closure,
    \item TRM and geometric dark energy arise naturally from the same source.
\end{itemize}

The next section assembles these ingredients into a minimal,
observationally testable TAP cosmology.

% ============================================================
% IX. MINIMAL TAP COSMOLOGY: BACKGROUND EVOLUTION
% ============================================================
\section{Minimal TAP cosmology: background evolution}
\label{sec:minimalTAP}

Having assembled the TAP geometric sectors and the asymmetry scalar
$A(a)$, we now formulate the minimal TAP cosmology. This model reduces
to $\Lambda$CDM when the triadic residue vanishes ($A=0$), but departs
predictively from it once $A(a)$ grows near the onset of acceleration.
The background dynamics are governed by three TAP contributions:
\begin{enumerate}[leftmargin=*]
    \item a curvature-dependent effective Newton coupling
          $G_{\mathrm{eff}}(\rho)=G[1+\alpha A(a)]$,
    \item a rigidity-matter component with scaling
          $\rho_{\mathrm{TRM}} \propto a^{-3(1+\beta)} A(a)$,
    \item a geometric dark-energy term $\rho_\Lambda(a)\propto A(a)$.
\end{enumerate}

% ------------------------------------------------------------
\subsection{Modified Friedmann equations}

The Friedmann equation with TAP corrections takes the form
\begin{equation}
H^2(a)
= \frac{8\pi G_{\mathrm{eff}}(a)}{3}
\Big[
\rho_b a^{-3}
+ \rho_r a^{-4}
+ \rho_{\mathrm{TRM}}(a)
+ \rho_\Lambda(a)
\Big],
\label{eq:Friedmann_TAP}
\end{equation}
where $\rho_b$ and $\rho_r$ denote the present-day baryon and photon
densities, respectively. The effective Newton coupling is
\begin{equation}
G_{\mathrm{eff}}(a)
= G\left[1 + \alpha\,A(a)\right].
\label{eq:Geff}
\end{equation}

The acceleration equation becomes
\begin{equation}
\frac{\ddot{a}}{a}
= -\frac{4\pi G_{\mathrm{eff}}(a)}{3}
\big(\rho + 3p\big)
+ \frac{4\pi G}{3}\left[\lambda_\Lambda A(a)\right],
\label{eq:accel_TAP}
\end{equation}
where $\lambda_\Lambda$ sets the normalization of the geometric
triadic dark-energy sector.

% ------------------------------------------------------------
\subsection{TRM density evolution}

The TAP-induced rigidity-matter component obeys
\begin{equation}
\rho_{\mathrm{TRM}}(a)
= \rho_{\mathrm{TRM},0}\,a^{-3(1+\beta)}\,A(a),
\label{eq:rhoTRM}
\end{equation}
where $\beta$ controls the stiffness of the rigidity sector.
For $\beta=0$ it behaves as dust; for $\beta>0$ it decays slightly more
slowly than CDM. Typical TAP phenomenology uses
$0.05 \lesssim \beta \lesssim 0.2$.

Because $A(a)\ll 1$ during radiation and matter domination,
$\rho_{\mathrm{TRM}}$ is dynamically irrelevant at early times and only
activates near the acceleration epoch.

% ------------------------------------------------------------
\subsection{Geometric dark-energy sector}

The TAP dark-energy density is proportional to the asymmetry scalar:
\begin{equation}
\rho_\Lambda(a)
= \rho_{\Lambda,0}\,\frac{A(a)}{A_0},
\label{eq:rhoLambdaA}
\end{equation}
where $A_0 \equiv A(a=1)$. This ensures that $\rho_\Lambda(a)$ tracks
the late-time growth and saturation of $A(a)$, producing a mildly
evolving effective equation of state.

To leading order, the dark-energy equation of state is
\begin{equation}
w_\Lambda(a)
= -1 + \frac{1}{3}\frac{d\ln A(a)}{d\ln a}.
\label{eq:wLambda}
\end{equation}

Since $A(a)$ increases as the Universe exits matter domination,
$w_\Lambda(a)$ naturally deviates slightly from $-1$ at intermediate
redshift, consistent with current DESI sensitivity windows.

% ------------------------------------------------------------
\subsection{Background solution and observational window}

Combining
\cref{eq:Friedmann_TAP,eq:Geff,eq:rhoTRM,eq:rhoLambdaA},
the Hubble parameter is
\begin{align}
H^2(a)
&=
H_0^2
\bigg[
\Omega_b a^{-3}
+ \Omega_r a^{-4}
+ \Omega_{\mathrm{TRM}}\,a^{-3(1+\beta)}A(a) \nonumber \\
&\quad\;
+ \Omega_{\Lambda}\frac{A(a)}{A_0}
\bigg]
\left[1+\alpha A(a)\right].
\label{eq:H2final}
\end{align}

The structure of \cref{eq:H2final} implies:
\begin{itemize}[leftmargin=*]
    \item TAP corrections are negligible at early times because  
          $A(a)\propto a^{-3/2}$ or $a^{-2}$.
    \item TAP modifies only the late-time expansion, exactly where
          current observations show mild anomalies (Hubble tension,
          DESI $w(z)$ evolution).
    \item TAP predicts gentle deviations in $H(a)$ for
          $0.3 \lesssim z \lesssim 2$, the prime sensitivity region of
          DESI, Euclid, Roman, and LSST.
\end{itemize}

% ------------------------------------------------------------
\subsection{Connection to the observed cosmic energy partition}

Evaluating \cref{eq:H2final} at $a=1$ yields the constraint
\begin{equation}
\Omega_b + \Omega_r
+ \Omega_{\mathrm{TRM}} A_0
+ \Omega_\Lambda = 1.
\label{eq:closure}
\end{equation}

The TAP phenomenology that reproduces the observed
$(\Omega_b,\Omega_{\mathrm{DM}},\Omega_\Lambda)
\approx (0.05,0.27,0.68)$ typically corresponds to
\begin{align}
A_0 &\sim 0.2, \\
\Omega_{\mathrm{TRM}}A_0 &\approx 0.27, \\
\Omega_\Lambda &\approx 0.68.
\end{align}

Thus the entire dark sector arises from two TAP sources:
\begin{itemize}[leftmargin=*]
    \item TRM provides the effective dark-matter fraction,
    \item geometric non-closure $A(a)$ provides dark energy.
\end{itemize}

There are no new particles and no new fields beyond geometry.

% ------------------------------------------------------------
\subsection{Summary of predictions}

The minimal TAP cosmology predicts:
\begin{itemize}[leftmargin=*]
    \item a mild deviation from $w=-1$ at $0.5\lesssim z\lesssim 2$,
    \item a density-stiffened $\rho_{\mathrm{TRM}}$ component mimicking CDM,
    \item a late-time shift in $G_{\mathrm{eff}}$ at $\mathcal{O}(10\%)$ level,
    \item no modifications to early-Universe physics,
    \item unified geometric origin of DM + DE.
\end{itemize}

In the next section we quantify TAP’s linear-perturbation signatures and
its observational discriminants.

% ============================================================
% X. LINEAR PERTURBATIONS AND OBSERVATIONAL SIGNATURES
% ============================================================
\section{Linear perturbations and observational signatures}
\label{sec:perturbations}

The TAP framework modifies the late-time evolution of both the metric
and matter fluctuations through two channels: (1) the density-dependent
Newton coupling $G_{\mathrm{eff}}(\rho)$, and (2) the rigidity-matter
(TRM) sector scaling with $A(a)$. Early-Universe physics remains
unchanged because $A(a)\ll 1$ in the radiation and matter eras.
Consequently, TAP preserves the CMB acoustic peaks, primordial
nucleosynthesis, and early growth while generating mild deviations only
at $z\lesssim 3$.

% ------------------------------------------------------------
\subsection{Modified Poisson equation}

In Newtonian gauge,
\begin{equation}
\nabla^2\Phi = 4\pi G_{\mathrm{eff}}(a)\,a^2
\left[
\rho_b\delta_b
+ \rho_{\mathrm{TRM}}\delta_{\mathrm{TRM}}
\right],
\label{eq:Poisson}
\end{equation}
where $\Phi$ is the gravitational potential and
$G_{\mathrm{eff}}(a)=G[1+\alpha A(a)]$. Because $A(a)$ increases only at
late times, the Poisson equation approaches GR at high redshift and
deviates gently for $z\lesssim 2$.

The TRM perturbation satisfies
\begin{equation}
\delta_{\mathrm{TRM}}' =
-(1+\beta)\theta_{\mathrm{TRM}}
- 3\frac{A'}{A}\delta_{\mathrm{TRM}},
\end{equation}
reflecting its slightly stiffened effective equation of state.

% ------------------------------------------------------------
\subsection{Growth equation}

For sub-horizon modes the baryon density contrast obeys
\begin{align}
\delta_b'' 
&+ \left(2 + \frac{H'}{H} \right)\delta_b'
\nonumber \\
&= \frac{3}{2}
\frac{H_0^2}{H^2(a)}
\,\Omega_{m}(a)\,\big[1+\alpha A(a)\big]\,\delta_b,
\label{eq:growth}
\end{align}
where primes denote derivatives with respect to $\ln a$.
The factor $1+\alpha A(a)$ provides a late-time enhancement or
suppression depending on the sign of $\alpha$. Because TAP phenomenology
matches cosmic screening to GR, we typically consider $\alpha>0$ but
small ($\alpha\sim0.1$--$0.2$).

The observable growth rate is
\begin{equation}
f(a) \equiv \frac{d\ln \delta_b}{d\ln a},
\qquad
f\sigma_8(a) = f(a)\,\sigma_8(a).
\label{eq:fsigma8}
\end{equation}

TAP generically predicts a few-percent deviation in $f\sigma_8$ at
$0.5\lesssim z\lesssim 2$, within DESI and Euclid sensitivity.

% ------------------------------------------------------------
\subsection{Gravitational slip}

TAP does not introduce additional degrees of freedom, so the two metric
potentials $\Phi$ and $\Psi$ remain equal at the linear level:
\begin{equation}
\Phi = \Psi.
\end{equation}
This is an important discriminant: many scalar–tensor and modified gravity frameworks\cite{Kase2019HorndeskiBeyond,Sotiriou2010fRreview})
predict $\Phi\neq\Psi$, whereas TAP predicts a \emph{purely geometric}
shift in $G_{\mathrm{eff}}$ without new fields.

% ------------------------------------------------------------
\subsection{Integrated Sachs--Wolfe (ISW) effect}

Because $A(a)$ increases only in the low-redshift Universe, the
gravitational potentials begin to evolve slightly at $z\lesssim 2$,
leading to a small but non-zero ISW signal. The magnitude is comparable
to a mildly dynamic dark-energy model with $w(z)\gtrsim -1$, making TAP
ISW predictions distinguishable from $\Lambda$CDM but consistent with
current CMB-LSS cross-correlation data.

% ------------------------------------------------------------
\subsection{Redshift-space distortions}

The TAP growth modification \cref{eq:growth} implies changes in peculiar
velocities and thus redshift-space distortions (RSD). Surveys measuring
$f\sigma_8$, including DESI, Euclid, and LSST, will be able to test TAP
through the predicted $\sim$1--5\% deviation at intermediate redshift.

% ------------------------------------------------------------
\subsection{Distinguishing TAP from $\Lambda$CDM}

The following late-time signatures jointly distinguish TAP:
\begin{itemize}[leftmargin=*]
    \item \textbf{Mild evolution of $w(z)$}, governed by $A(a)$.
    \item \textbf{Enhanced or suppressed growth} depending on $G_{\mathrm{eff}}(a)$.
    \item \textbf{CDM-like but slightly stiff TRM} modifying the matter budget.
    \item \textbf{No gravitational slip}, unlike scalar-tensor or MG theories.
    \item \textbf{No early-time deviations}: TAP preserves CMB and BAO.
\end{itemize}

Combined, these produce a testable and tightly constrained deviation
from $\Lambda$CDM that will be decidable with near-future data.

% ============================================================
% XI. DISCUSSION AND OUTLOOK
% ============================================================
\section{Discussion and outlook}
\label{sec:discussion}

The Triadic Asymmetry Principle provides a unified geometric mechanism
in which motion, time, and space form a non-commuting ordered triad.
The resulting triadic residue $A=\|[\M,\T,\Sop]\|$ acts as a single
geometric driver of three late-time cosmological channels:
density-modulated gravity through $G_{\mathrm{eff}}(\rho)$, rigidity-
dominated matter (TRM) scaling as $a^{-3(1+\beta)}$, and a geometric
dark-energy sector with $\rho_\Lambda\propto A(a)$. Because $A(a)$ is
naturally suppressed in high-density or high-curvature environments, TAP
automatically reproduces GR for all Solar-System and astrophysical
tests, while permitting percent-level modifications on cosmic scales.

A central feature of TAP is that no new fields, potentials, or
fine-tuned parameters are introduced. The deformation of gravity arises
solely from the internal non-closure of the $(\M,\T,\Sop)$ triad, and
the late-time behavior of $A(a)$ determines both the geometry and the
expansion history. As a result, several long-standing cosmological
phenomena emerge from a single geometric source.

\subsection{Implications for cosmic acceleration}

Because the geometric dark-energy density tracks the growth of
$A(a)$, the expansion history exhibits an effective equation of state
$w(z)>-1$ at intermediate redshift but closely approaches $\Lambda$CDM
today. This mild dynamical component provides a natural explanation for
the DESI preference for evolving dark energy \cite{Linder2003CPL}without invoking additional
scalar fields or exotic sectors.

\subsection{Structure formation and near-future tests}

TAP predicts specific signatures in the growth rate, RSD observables,
and the ISW effect. The predicted $1$–$5$\,\% deviation in $f\sigma_8$
at $0.5\lesssim z\lesssim 2$ lies squarely within the sensitivity of
DESI, Euclid, and the Rubin Observatory. The absence of gravitational
slip ($\Phi=\Psi$) provides a clean test distinguishing TAP from
scalar-tensor theories and many modified-gravity alternatives.

\subsection{Relation to other frameworks}

TAP differs from modified-gravity models by introducing no new
propagating degrees of freedom, and from dynamical dark-energy models by
generating $w(z)$ geometrically rather than through a potential.
Similarly, the TRM sector behaves like CDM with a slight rigidity
correction, distinct from warm dark matter or self-interacting dark
matter scenarios. In combination, these traits place TAP in a unique
position among post-$\Lambda$CDM approaches.

\subsection{Future directions}

There are several natural extensions of the present work:

\begin{itemize}[leftmargin=*]
    \item \textbf{Nonlinear structure formation.}  
    Extending TAP to the nonlinear regime, including halo formation and
    N-body simulations, will assess TRM clustering and potential small-
    scale signatures.

    \item \textbf{Relativistic effects.}  
    A full relativistic perturbation analysis incorporating light
    propagation, lensing convergence, and cosmic shear will enable
    comparison to weak-lensing datasets.

    \item \textbf{Bounce cosmology and the antiverse.}  
    TAP provides a natural language for dual time-flow cosmologies.
    Future work can develop the triadic exchange across a bounce and the
    relation between the Universe–Antiverse dyad.

    \item \textbf{Quantum and geometric origin of the triad.}  
    A deeper investigation into whether the $(\M,\T,\Sop)$ operators can
    emerge from a more fundamental micro-geometric or algebraic
    structure—such as causal sets, CDT, or operator-algebra quantum
    gravity—may reveal the origin of triadic asymmetry.
\end{itemize}

\vspace{6pt}
In summary, the Triadic Asymmetry Principle provides a tightly
constrained, observationally testable geometric extension of GR in which
dark matter, dark energy, and density-modulated gravity arise from a
single structural feature: the ordered non-commutativity of
motion, time, and space. The upcoming decade of cosmological surveys
will decisively test this framework.

\subsection*{Limitations and Assumptions}

The present formulation of TAP makes several simplifying assumptions that 
define its current scope and highlight directions for future development. 
First, all results are derived at the background and linear-perturbation 
levels; a complete nonlinear analysis, including halo formation and 
N-body dynamics with TRM, is left for future work. Second, the operator 
representation of $(\M,\T,\Sop)$ used here is chosen for cosmological 
symmetry; while different representations yield equivalent large-scale 
phenomenology, they may shift the normalization of the asymmetry scalar 
$\A$, and exploring this dependence is an open task. Third, TAP assumes 
statistical homogeneity and isotropy on cosmological scales, and does 
not address potential signatures in strongly anisotropic or inhomogeneous 
backgrounds. Finally, we treat the deformation function $F(\A)$ 
phenomenologically, specifying only its regularity and small-$\A$ 
expansion; determining whether $F(\A)$ arises from a more fundamental 
algebraic or microgeometric principle is an ongoing direction. These 
assumptions do not affect the main results but clarify the regime in 
which the present analysis is valid.

% ============================================================
% XII. APPENDICES
% ============================================================

\appendix

% ------------------------------------------------------------
% Appendix A: Derivation of the TAP-modified field equations
% ------------------------------------------------------------
\section{Derivation of the TAP-modified field equations}
\label{app:derivation}

The TAP deformation enters the gravitational action through a modified
volume measure,
\begin{equation}
    S = \frac{1}{16\pi G}
    \int d^4x\, \sqrt{-g}\, F(A)\, R + S_{\rm m},
\end{equation}
with $F(0)=1$ and $A=\|[\M,\T,\Sop]\|$. Varying with respect to the
metric gives
\begin{equation}
    \delta S =
    \frac{1}{16\pi G}
    \int d^4x
    \bigl[
      \delta(\sqrt{-g}F(A))\, R
      + \sqrt{-g}F(A)\, \delta R
    \bigr].
\end{equation}

Using the metric variation
\begin{equation}
    \delta(\sqrt{-g}) = -\tfrac12\sqrt{-g} g_{\mu\nu}\delta g^{\mu\nu},
\end{equation}
and defining
\begin{equation}
    F_A \equiv \frac{dF}{dA},
\end{equation}
we obtain
\begin{equation}
\delta(\sqrt{-g}F) =
    \sqrt{-g}\left(
      -\frac{1}{2}Fg_{\mu\nu}\delta g^{\mu\nu}
      + F_A\,\delta A
    \right).
\end{equation}

Variation of the Ricci scalar yields the standard Einstein–Hilbert
boundary structure. Collecting all pieces produces the TAP-modified
field equations,
\begin{equation}
    F(A)G_{\mu\nu}
    + \bigl(g_{\mu\nu}\Box - \nabla_\mu\nabla_\nu\bigr)F(A)
    = 8\pi G\, T_{\mu\nu},
    \label{eq:TAP-field-eq-appendix}
\end{equation}
which reduce exactly to GR when $A\rightarrow 0$.

In FRW symmetry, the operator-alteration structure constrains $A$ to be
a function of the scale factor alone, $A=A(a)$, simplifying the
derivatives and yielding the Friedmann equations used in the main text.

% ------------------------------------------------------------
% Appendix B: TRM scaling and rigidity parameter beta
% ------------------------------------------------------------
\section{TRM scaling and the rigidity parameter \texorpdfstring{$\beta$}{beta}}
\label{app:TRM}

The rigidity-dominated matter component (TRM) arises from the expansion
of the modified measure $F(A)$ around $A=0$:
\begin{equation}
    F(A) = 1 + \alpha A + \gamma A^2 + \cdots.
\end{equation}

The linear deformation induces a correction to the continuity equation,
\begin{equation}
    \dot{\rho} + 3H\rho
    = -3H\beta\rho,
\end{equation}
where the parameter $\beta$ is directly linked to the response of $A$ to
cosmic expansion,
\begin{equation}
    \beta = \frac{d\ln A}{d\ln a}.
\end{equation}

Solving the modified continuity equation yields
\begin{equation}
    \rho_{\mathrm{TRM}}
    = \rho_0\, a^{-3(1+\beta)},
\end{equation}
the result quoted in the main text.

Unlike warm or self-interacting dark matter, TRM does not correspond to
a new particle species. It is a geometric rigidity effect arising from
the ordered non-commutativity of $(\M,\T,\Sop)$ and exhibits cold-matter
clustering on large scales with a small, calculable deviation controlled
by $\beta$.

% ------------------------------------------------------------
% Appendix C: Operator locking and high-curvature screening
% ------------------------------------------------------------
\section{Operator locking and high-curvature screening}
\label{app:screening}

In regions where curvature or matter density dominates over cosmological
triadic structure, the geometric operators tend to align:
\begin{equation}
    [\M,\T,\Sop]\;\longrightarrow\;0,
    \qquad
    \text{for } R_{\mu\nu\alpha\beta}\gg H^2,\,\dot{H}.
\end{equation}

Since every TAP deformation is proportional to $A$, the $A\to0$ limit
restores the Einstein–Hilbert action:
\begin{equation}
    S \to \frac{1}{16\pi G} \int d^4x\, \sqrt{-g}\, R.
\end{equation}

This provides a purely geometric screening mechanism. Dense regions,
collapsed objects, and high-curvature environments reproduce GR to high
accuracy:
\begin{equation}
    A_{\rm local} \lesssim 10^{-15}-10^{-20},
\end{equation}
ensuring consistency with Solar System tests, binary pulsars, and
precision timing constraints.

The TAP deformation therefore acts only on cosmological scales, while remaining invisible in all presently tested
local-gravity regimes.

\section*{Acknowledgments}

The author thanks colleagues in the theoretical cosmology and gravitation
communities for valuable discussions and early feedback on the conceptual
development of the Triadic Asymmetry Principle. The author is also grateful
for the support of the broader online scientific community, whose questions
and insights helped sharpen several aspects of the framework. Finally, the
author thanks his wife Cindy for her patience in tolerating many late-night hours spent developing this framework, which for a while seemed like a ridiculous idea — and a total waste of motion, time, and space. Any remaining errors are the author's own.

% Print *all* entries in tap_refs.bib, even if not cited explicitly:
\nocite{*}

\bibliographystyle{apsrev4-2}
\bibliography{tap_refs}

\end{document}